% (The MIT License)
%
% Copyright (c) 2023 Yegor Bugayenko
%
% Permission is hereby granted, free of charge, to any person obtaining a copy
% of this software and associated documentation files (the 'Software'), to deal
% in the Software without restriction, including without limitation the rights
% to use, copy, modify, merge, publish, distribute, sublicense, and/or sell
% copies of the Software, and to permit persons to whom the Software is
% furnished to do so, subject to the following conditions:
%
% The above copyright notice and this permission notice shall be included in all
% copies or substantial portions of the Software.
%
% THE SOFTWARE IS PROVIDED 'AS IS', WITHOUT WARRANTY OF ANY KIND, EXPRESS OR
% IMPLIED, INCLUDING BUT NOT LIMITED TO THE WARRANTIES OF MERCHANTABILITY,
% FITNESS FOR A PARTICULAR PURPOSE AND NONINFRINGEMENT. IN NO EVENT SHALL THE
% AUTHORS OR COPYRIGHT HOLDERS BE LIABLE FOR ANY CLAIM, DAMAGES OR OTHER
% LIABILITY, WHETHER IN AN ACTION OF CONTRACT, TORT OR OTHERWISE, ARISING FROM,
% OUT OF OR IN CONNECTION WITH THE SOFTWARE OR THE USE OR OTHER DEALINGS IN THE
% SOFTWARE.

\documentclass{article}
\usepackage{clicks}
\usepackage{../../lecture-notes/notes}
\usepackage{svg}

\graphicspath{{../../faces/}}

\newcommand*{\thetitle}{Coders vs. Researchers}
\newcommand*{\thesubtitle}{How to be both?}
\newcommand*{\theauthor}{Yegor Bugayenko}

\begin{document}

\pptLeft{%
  \includesvg[height=2em]{../../lecture-notes/logos/innopolis.svg}\\
  November 23th, 2024}
\pptRight{@yegor256}

\definecolor{innoBackground}{HTML}{16240E}
\pagecolor{innoBackground}
\plush{
  \begin{pptMiddle}
    \includesvg[height=4em]{logo.svg}
    \pptTitle{\thetitle}{\thesubtitle}\par
    {\color{white}\scshape \theauthor}
    \newline
    {\color{white}\small Zerocracy\par}
  \end{pptMiddle}
}
\newpage
\pagecolor{white}

\lnThought*{What exactly is R\&D?}

\lnQuote
  [Richard Nelson]
  {richard-nelson}
  {Scientific \textcolor{orange}{research} may be defined as the human activity directed toward the advancement of knowledge, where knowledge is of two roughly separable sorts: \begin{inparaenum}[1)]\item \ul{facts or data} observed in reproducible experiments and
  \item \ul{theories or relationships} between data\end{inparaenum}.}
  {nelson1959simple}

\lnThought*{While research discovers laws of nature, development exploits them.}

\plick{
  \begin{textblock}{4}(1,3)%
    Task: ``Implement a lookup function, capable of finding user IDs by their names.''\par
    {\ttfamily\scriptsize
    use std::collections::HashMap;\\
    ~\\
    let mut map: Map<String, i16> = \\
    ~~HashMap::new();\\
    ~\\
    map.insert("Jeff".to_string(), 42);\\
    map.insert("Lucie".to_string(), 17);\\
    // +N more items here\\
    ~\\
    let id = map.get("Jeff");
    }
  \end{textblock}%
}
\plick{
  \begin{textblock}{4}(6,3)%
    \textcolor{orange}{RQ}: ``Whether HashMap is the best possible map implementation for \(N < 20\)?''\par
    Answer:\par

  \end{textblock}%
}
\plush{
  \begin{textblock}{4}(11,3)%
    New \textcolor{orange}{implementation}, based on the \textcolor{orange}{law of nature} just discovered:\par
    {\ttfamily\scriptsize
    use std::collections::HashMap;\\
    use micromap::Map;\\
    ~\\
    let mut map: Map<String, i16>;\\
    if (N > 20) \{ \\
    ~~map = HashMap::new();\\
    \} else \{\\
    ~~map = Map::new();\\
    \}\\
    ~\\
    map.insert("Jeff".to_string(), 42);\\
    map.insert("Lucie".to_string(), 17);\\
    // +N more items here\\
    ~\\
    let id = m.get("Jeff");
    }
  \end{textblock}%
}
\flush{}

\lnThought*{What's in it for you?}

\lnQuote
  [Aristotle]
  {aristotle}
  {Since happiness is a \ul{working} in the way of excellence, the man of science will be most happy.}
  {aristotle1897nicomachean}

% list of results to be achieved, ending with Hirsh index

\plush{
  \includegraphics[width=.9\linewidth]{openai.jpg}\par
  {\small20-Sep-24: \url{https://www.youtube.com/watch?v=tEzs3VHyBDM}\par}}

\lnThought*{Who judges the work of developers and researchers?}

\plush{
  \begin{multicols}{2}
  \pptBanner{Users:}\par
  \pptPic{.7}{customers.jpg}
  \par\columnbreak\par
  \pptBanner{Reviewers:}\par
  \pptPic{.7}{reviewers.jpg}
  \end{multicols}}

\lnThought*{``OK, I'm a programmer, what could be the possible \ul{Research Questions} for me?''}

% how much slower is Xembly
% which JVM is faster?
% how the usage of anti-patterns correlate with bugs/LoC?
%

\lnThought*{How do you convince your company to invest into a research project?}

\lnQuote
  [Kenneth Joseph Arrow]
  {kenneth-joseph-arrow}
  {We expect a free enterprise economy to \textcolor{orange}{underinvest} in invention and research (as compared with an \ul{ideal}) because it is risky, because the product can be appropriated only to a limited extent, and because of increasing returns in use. This underinvestment will be greater for more \ul{basic} research.}
  {arrow1972economic}

\lnQuote
  [Bronwyn Hall]
  {bronwyn-hall}
  {First and most importantly, in practice \textcolor{orange}{50\% or more} of R\&D spending is the \ul{wages and salaries} of highly educated \ul{scientists and engineers}. Their efforts create an intangible asset, the firm’s knowledge base, from which profits in future years will be generated. To the extent that this knowledge is `tacit' rather than codified, it is embedded in the human capital of the firm's employees, and is therefore lost if they \ul{leave} or are \ul{fired}.}
  {hall2002financing}

\lnThought*{What's next?}

\plush{
  \begin{multicols}{2}
  \pptBanner{1)~Subscribe:}\par
  \qrcode[height=8em]{https://t.me/yegor256news}\par
  \href{https://t.me/yegor256news}{\texttt{@yegor256news}}
  \par\columnbreak\par
  \pptBanner{2)~Join:}\par
  Text me in Telegram, if you're ready to join one of our research projects:\par
  {\Huge\texttt{@yegor256}}
  \end{multicols}}

\end{document}
