% SPDX-FileCopyrightText: Copyright (c) 2023 Yegor Bugayenko
% SPDX-License-Identifier: MIT

\documentclass{article}
\usepackage[T1]{fontenc}
\usepackage[utf8]{inputenc}
\usepackage[nocn,nonumbers,noframes,sf,bold]{ffcode}
\usepackage[static]{clicks}
\usepackage{tikzsymbols}
\usetikzlibrary{positioning}
\usepackage[increment]{crumbs}
\usepackage[template,scheme=light,nominutes]{ppt-slides}
\usepackage{href-ul}
\usepackage{svg}
\renewcommand{\ttdefault}{cmtt}
\newcommand\nospell[1]{#1}
\newcommand*{\thetitle}{Robots vs. Programmers}
\newcommand*{\thesubtitle}{How must time we have until they take over?}
\newcommand*{\theauthor}{Yegor Bugayenko}
\pptLeft{\includesvg[height=.6in]{innopolis-logo.svg}}
\pptRight{\includesvg[height=.6in]{aiin-logo.svg}}

\newcommand\personAndPhoto[3]{%
  \parbox{1in}{\raggedright%
    \includegraphics[height=1.2in]{#3}\par
    \scriptsize #1 \newline #2 \par
  }
}
\newcommand\person[2]{\personAndPhoto{#1}{#2}{#1-#2.jpg}}

\begin{document}

\plush{
  \begin{pptMiddle}
    \includesvg[height=1in]{aiin-logo.svg}
    \pptTitle{\thetitle}{\thesubtitle}\par
    {\scshape \theauthor}
    \newline
    {\small Huawei Technologies Co., Ltd.\par}
  \end{pptMiddle}
}

\pptToc

\plush{\pptChapter{Today}}

\pptSection[Complete]{Auto Code Completion}
This is how \href{https://github.com/features/copilot}{Copilot} by GitHub is suggesting code completion in our \href{https://www.eolang.org}{own programming language}, which it definitely hasn't seen before:
\par
\begin{multicols}{4}
\pptPic{.9}{copilot-1.png}
\pptPic{.9}{copilot-2.png}
\pptPic{.9}{copilot-3.png}
\pptPic{.9}{copilot-4.png}
\end{multicols}
\par
Read also about \href{https://aws.amazon.com/codewhisperer/}{AWS CodeWhisperer} at \href{https://www.allthingsdistributed.com/2023/04/how-ai-coding-companions-will-change-the-way-developers-work.html}{Werner Vogels' blog}. Also, about \href{https://github.com/codota}{TabNine} (used to be Codota).
\plush{}

\pptSection[PR]{Pull Request Explanation}
\pptPic{0.75}{pr-codex.png}
\par
\href{https://github.com/decentralizedlabs/pr-codex}{PR-Codex} plugin for GitHub by \href{https://www.dlabs.app/}{dlabs}
\plush{}

\pptSection[Review]{Reviewing Changes}
\pptPic{0.75}{codex-qa.png}
\par
The discussion happened in this GitHub issue: \href{https://github.com/objectionary/eo/pull/2034}{objectionary/eo:2034}
\plush{}

\pptSection[Risks]{Pull Request Risk Analysis}
\pptPic{0.9}{codeball.png}
\par
\href{https://codeball.ai/}{Codeball} plugin for GitHub
\plush{}

\pptSection[Explain]{Explain This to Me!}
\pptPic{0.9}{mintlify.png}
\par
\href{https://github.com/mintlify/writer}{Writer} plugin for VS-Code by \href{https://writer.mintlify.com/}{Mintlify}
\plush{}

\pptSection[Safurai]{All in One Package}
\pptPic{0.9}{safurai.png}
\par
Plugin for VS-Code by \href{https://www.safurai.com/}{Safurai}
\plush{}

\pptSection[Repeat]{Repeat After Me!}
\pptPic{0.6}{intellicode.png}
\par
Read about ``\href{https://devblogs.microsoft.com/visualstudio/making-repeated-edits-easier-with-intellicode-suggestions/}{making repeated edits easier with IntelliCode suggestions}'', by Peter Groenewegen.
\plush{}

\pptSection[Bugs]{Detecting Bugs}
\pptPic{.7}{catid.png}
\par
Can open-source LLMs detect bugs in C++ code?
\newline
\url{https://catid.io/posts/llm_bugs/}
\plush{}

\pptSection[OverflowAI]{OverflowAI}
\pptPic{.7}{overflowai.png}
\par
\url{https://stackoverflow.blog/2023/07/27/announcing-overflowai/}
\plush{}

\pptSection[PDD]{Puzzle Driven Development (PDD)}
\begin{pptWide}{2}
\resizebox{\columnwidth}{!}{\begin{tikzpicture}[
  every node/.style={draw=black, line width=.1em, font={\ttfamily},inner sep=.5em, outer sep=.2em},
  every path/.style={draw=black, line width=.1em},
]
\node[] (problem) {Problem};
\node[draw=none,below=2em of problem] (dev) {\parbox{5em}{\centering\Strichmaxerl[5] \\ Developer}};
\node[right=2em of dev] (solution) {Solution};
\node[draw=none,below=0cm of solution] {\parbox{8em}{\centering + more problems (aka "puzzles")}};
\node[right=4em of solution] (scanner) {\parbox{6em}{\centering Source Code Scanner}};
\node[below=5em of scanner] (tickets) {\parbox{6em}{\centering Ticket Tracking System}};
\node[draw=none,left=2em of tickets] (robot) {\parbox{3em}{\centering\Strichmaxerl[5] \\ PDD Robot}};
\path[->] (problem) -- (dev);
\path[->] (dev) -- (solution);
\path[->] (solution) -- (scanner);
\path[->] (scanner) -- (tickets);
\path[->] (tickets) -- (robot);
\path[->] (robot) edge[bend left=25,line width=.2em,draw=orange] node[fill=white,left=-1em] {\parbox{8em}{\centering New Prioritized Problems}} (dev);
\end{tikzpicture}}
\par\columnbreak\par
\person{Giancarlo}{Succi}
\person{Witold}{Pedrycz}
\person{Yaroslav}{Plaksin}
\person{Mirko}{Farina}
\person{Artem}{Kruglov}
\par
{\small Y.~Bugayenko, M.~Farina, A.~Kruglov, W.~Pedrycz, Y.~Plaksin, G.~Succi,
\textit{Automatically Prioritizing Tasks in Software Development}, IEEE Access, 2023}
\par
The robot is a GitHub chatbot: \url{www.0pdd.com}
\end{pptWide}
\plush{}

\plush{\pptChapter{Tomorrow}}

\pptSection[Review]{Review pull requests (Copilot X)}
\pptPic{.7}{copilot-pr.png}
\par
\url{https://githubnext.com/projects/copilot-for-pull-requests}
\plush{}

\pptSection[Tests]{Write unit tests (TestPilot)}
\pptPic{.9}{testpilot.png}
\par
\url{https://githubnext.com/projects/testpilot/}
\plush{}

\pptSection[Visualise]{Visualise code base (Copilot X)}
\pptPic{.8}{copilot-visualize.png}
\par
\url{https://githubnext.com/projects/repo-visualization/}
\plush{}

\pptSection[Refactor]{Automated Refactoring}
\begin{multicols}{2}
\pptPic{.9}{codescene.png}
\par\columnbreak\par
I found this picture in the \href{https://codescene.com/engineering-blog/refactoring-recommendations}{CodeScene website}.
\end{multicols}
\plush{}

\pptSection[CAPA]{Suggest Corrective and Preventive Actions}
\begin{pptWide}{2}
\resizebox{.7\columnwidth}{!}{\begin{tikzpicture}[
  every node/.style={draw=black, line width=.1em, font={\ttfamily},inner sep=.5em, outer sep=.2em},
  every path/.style={draw=black, line width=.1em},
]
\node[draw=none] (tom) {\parbox{4em}{\centering\Strichmaxerl[5] \\ TOM Robot w/ML}};
\node[above right=4em and 0em of tom] (metrics) {\parbox{5em}{\centering Code Metrics}};
\node[above left=2em of tom] (tickets) {\parbox{6em}{\centering Tickets}};
\node[above=6em of metrics] (scanners) {\parbox{6em}{\centering Source Code Scanners}};
\node[draw=none,above left=4em and 0em of tickets, anchor=base] (dev1) {\parbox{2em}{\centering\Strichmaxerl[5] \\ Dev}};
\node[draw=none,above right=4em and 0em of tickets, anchor=base] (dev2) {\parbox{2em}{\centering\Strichmaxerl[5] \\ Dev}};
\node[above=9em of tickets] (repo) {\parbox{6em}{\centering GitHub Repository}};
\path[->] (repo) -- (scanners);
\path[->] (scanners) -- (metrics);
\path[->] (metrics) -- (tom);
\path[->] (tom) edge[bend left=25,line width=.2em,draw=orange] node[fill=white,left=-1em] {CAPAs} (tickets);
\path[->] (tickets) -- (dev1);
\path[->] (tickets) -- (dev2);
\path[->] (dev1) -- (repo);
\path[->] (dev2) -- (repo);
\end{tikzpicture}}
\par\columnbreak\par
\person{Giancarlo}{Succi}
\person{Witold}{Pedrycz}
\person{Mirko}{Farina}
\person{Artem}{Kruglov}
\\
\person{Zamira}{Kholmatova}
\person{Firas}{Jolha}
\person{Arina}{Cheverda}
\person{Kirill}{Daniakin}
\par
{\scriptsize Y.~Bugayenko, K.~Daniakin, M.~Farina, F.~Jolha, A.~Kruglov, G.~Succi, W.~Pedrycz,
\textit{Extracting Corrective Actions from Code Repositories},
Proceedings of the 19th International Conference on Mining Software Repositories (MSR), 2022\par}
\end{pptWide}
\plush{}

\plush{\pptChapter[Years]{In a Few Years}}

\pptSection[Reject]{Reject builds for suspicious quality}
\pptThought{A combination of non-deterministic (A) and deterministic (symbolic execution)  methods of quality control may be the future of static analysis}
\plush{}

\pptSection[Fix]{Make Pull Requests}
\pptThought{Bug fixes, new features, and refactorings may be coming directly from the AI to the project team, as pull requests}
\plush{}

\pptSection[Appraise]{Appraise our work}
\pptThought{Currently subjective appraisal decisions of humans may be replaced by more grounded decisions made by the AI}
\plush{}

\plush{\pptChapter[Decades]{In a Few Decades}}

\pptBanner{The robots...}
\pptPin{\raggedright%
  \pptPic{.95}{exmachina.jpg} \\
  \small Ex Machina (2014) by Alex Garland
}
\par
\print{They will tell us what to do}
\print{They will hire and fire us}
\print{They will decide how much to pay us}
\print{They will plan our \st{life} projects}
\plush{\small ... especially if they know how do we feel}

\end{document}
