% SPDX-FileCopyrightText: Copyright (c) 2022 Yegor Bugayenko
% SPDX-License-Identifier: MIT

\documentclass{article}
\usepackage[T1]{fontenc}
\usepackage[utf8]{inputenc}
\usepackage[nocn,nonumbers,noframes,sf,bold]{ffcode}
\usepackage{lmodern}
\usepackage{clicks}
\usepackage{setspace}
\usepackage{href-ul}
\usepackage{titling}
\usepackage{mathtools}
\renewcommand{\ttdefault}{cmtt}
\usepackage[template,scheme=light,nominutes]{ppt-slides}
\newcommand\nospell[1]{#1}
\title{Exceptions as Containers}
\newcommand*{\thesubtitle}{EOLANG Implementation Example}
\author{Yegor Bugayenko}
\pptLeft{\thetitle{}}
\pptRight{\theauthor}
\begin{document}

\plush{
  \begin{pptMiddle}
    \pptTitle{\thetitle}{\thesubtitle}\par
    {\scshape \theauthor}
  \end{pptMiddle}
}

\pptHeader{Objects and Their Names in EOLANG}
\begin{pptWide}{2}
``Two plus two'':
\begin{ffcode*}{fontseries=bf}
2.plus 2 > a
\end{ffcode*}
Similar to Pascal (almost):
\begin{ffcode*}{fontseries=bf}
var a : integer;
a := 2 + 2;
\end{ffcode*}
\columnbreak
Similar to Java (almost):
\begin{ffcode*}{fontseries=bf}
Integer a = new Integer(2)
  .plus(new Integer(2));
\end{ffcode*}
Alternative ``vertical'' syntax in EO:
\begin{ffcode*}{fontseries=bf}
plus. > x
  2
  2
\end{ffcode*}
\end{pptWide}

\plush{}
\pptHeader{Fibonacci in EOLANG}
\begin{pptWide}{2}
Canonical syntax:
\begin{ffcode*}{fontseries=bf}
[n] > fibo
  if. > @
    $.n.gt 2
    n
    plus.
      fibo (n.minus 1)
      fibo (n.minus 2)
\end{ffcode*}
\columnbreak
Alternative syntax:
\begin{ffcode*}{fontseries=bf}
[n] > fibo
  (n.gt 2).if > @
    n
    fibo (n.minus 1)
    .plus (fibo (n.minus 2))
\end{ffcode*}
Now, we can make a copy of the abstract object and then ``dataize'' it:
\begin{ffcode*}{fontseries=bf}
fibo 13 > x
print x
\end{ffcode*}
\end{pptWide}

\plush{}
\pptHeader{The Error Object}
\begin{pptWide}{2}
Instead of throwing an exception we return a copy of the |error| object:
\begin{ffcode*}{fontseries=bf}
[x] > foo
  if. > @
    x.eq 0
    error "Can't divide by zero!"
    42.div x
\end{ffcode*}
\columnbreak
Later:
\begin{ffcode*}{fontseries=bf}
(foo 0).plus 1 > x
\end{ffcode*}
Even later:
\begin{ffcode*}{fontseries=bf}
+alias org.eolang.io.stdout
+alias org.eolang.txt.sprintf

stdout
  sprintf
    "The result is %d"
    x
\end{ffcode*}
\end{pptWide}

\plush{}
\pptHeader{The Try Object}
\begin{pptWide}{2}
The dataization of the |error| object leads to a runtime error and immediate program termination, however it may be ``caught'' by the |try| object:
\begin{ffcode*}{fontsize=\scriptsize,fontseries=bf}
QQ.io.stdout
  try
    QQ.txt.sprintf
      "The result is %d"
      x
    [e]
      QQ.txt.sprintf > @
        "Oops: %s"
        e
    nop
\end{ffcode*}
\columnbreak
This code would be similar to the following code in Java (almost):
\begin{ffcode*}{fontsize=\scriptsize,fontseries=bf}
System.out.println(
  try {
    String.format(
      "The result is %d", x()
    )
  } catch (Exception e) {
    String.format(
      "Oops: %s", e.getMessage()
    )
  } finally {
    // nothing
  }
);
\end{ffcode*}
\end{pptWide}

\plush{}
\pptHeader{The Implementation}
\begin{pptWide}{2}
The following EO code:
\begin{ffcode*}{fontsize=\scriptsize,fontseries=bf}
error "Can't divide by zero!"
\end{ffcode*}
Is compiled to the following pseudo-Java:
\begin{ffcode*}{fontsize=\scriptsize,fontseries=bf}
class EOerror {
  private theEnclosure;
  dataize() {
    // Instead of "return data;"
    throw new ExError(
      theEnclosure
      // "Can't divide by zero!"
    );
  }
}
\end{ffcode*}
\columnbreak
Then, the |try| object is compiled to:
\begin{ffcode*}{fontsize=\scriptsize,fontseries=bf}
class EOtry {
  private theBody;
  private theCatch;
  private theFinally;
  dataize() {
    try {
      return dataize(theBody)
    } catch (ExError e) {
      return dataize(
        theCatch(e.getEnclosure())
      );
    } finally {
      return dataize(theFinally));
    }
  }
}
\end{ffcode*}
\end{pptWide}

\plush{}

\plush{
  \pptHeader{Check It Out on GitHub:}
  \url{https://github.com/objectionary/eo}\par
  \pptQR{https://github.com/objectionary/eo}\par
  ... and give it a star! :)
}

\end{document}
