% SPDX-FileCopyrightText: Copyright (c) 2022 Yegor Bugayenko
% SPDX-License-Identifier: MIT

\documentclass{article}
\usepackage[T1]{fontenc}
\usepackage[utf8]{inputenc}
\usepackage[nocn,nonumbers]{ffcode}
\usepackage{clicks}
\usepackage{setspace}
\usepackage{href-ul}
\usepackage{titling}
\usepackage{mathtools}
\renewcommand{\ttdefault}{cmtt}
\usepackage[template,scheme=light,nominutes]{ppt-slides}
\newcommand\nospell[1]{#1}
\title{Immutability + Generics}
\newcommand*{\thesubtitle}{An Immutable Collection of Strategies for EO Compiler}
\author{Yegor Bugayenko}
\pptLeft{\thetitle{}}
\pptRight{\theauthor}
\begin{document}

\plush{
  \pptMiddle{
    \pptTitle{\thetitle}{\thesubtitle}\par
    {\scshape \theauthor}\par
    \pptQR{https://www.yegor256.com/2022/08/10/xsline-immutable-pipeline.html}
  }
}

\pptHeader{What is DOM?}
\begin{pptWide}{2}
This is XML:
\begin{ffcode*}{fontsize=\scriptsize,fontseries=bf}
<garage>
  <car country="DE">BMW</car>
  <car country="SE">Volvo</car>
  <car country="DE">Audi</car>
  <car country="JP">Toyota</car>
</garage>
\end{ffcode*}
This is the result:
\begin{ffcode*}{fontsize=\scriptsize}
<garage DE="2">
  <car country="DE">BMW</car>
  <car country="SE">Volvo</car>
  <car country="DE">Audi</car>
  <car country="JP">Toyota</car>
</garage>
\end{ffcode*}
\columnbreak
This is DOM API of Java:
\begin{ffcode*}{fontsize=\scriptsize}
var garage = dom.getOwnerDocument()
  .getDocumentElement();
var count = 0;
var cars = garage.getChildNodes();
for (int i; i < cars.getLength(); ++i) {
  if (cars.item(i).getAttributes()
    .getNamedItem("country")
    .getTextContent().equals("DE")) {
    ++count;
  }
}
garage.setAttribute("total",
  Integer.toString(count));
\end{ffcode*}
\end{pptWide}

\plush{}
\pptHeader{What is XSL?}
\begin{pptWide}{2}
This is XML:
\begin{ffcode*}{fontsize=\scriptsize}
<garage>
  <car country="DE">BMW</car>
  <car country="SE">Volvo</car>
  <car country="DE">Audi</car>
  <car country="JP">Toyota</car>
</garage>
\end{ffcode*}
This is the result of XSL ``transformation'':
\begin{ffcode*}{fontsize=\scriptsize}
<garage DE="2">
  <car country="DE">BMW</car>
  <car country="SE">Volvo</car>
  <car country="DE">Audi</car>
  <car country="JP">Toyota</car>
</garage>
\end{ffcode*}
\columnbreak
This is XSL ``stylesheet'':
\begin{ffcode*}{fontsize=\scriptsize}
<xsl:stylesheet>
  <xsl:template match="/garage">
    <xsl:copy>
      <xsl:attribute name="DE"
        select="count(car[country='DE'])"/>
      <xsl:apply-templates
        select="node()"/>
    </xsl:copy>
  </xsl:template>
</xsl:stylesheet>
\end{ffcode*}
\end{pptWide}

\plush{}
\pptHeader{What is Xembly?}
\begin{pptWide}{2}
This is XML:
\begin{ffcode*}{fontsize=\scriptsize}
<garage>
  <car country="DE">BMW</car>
  <car country="SE">Volvo</car>
  <car country="DE">Audi</car>
  <car country="JP">Toyota</car>
</garage>
\end{ffcode*}
This is the result of Xembly evaluation:
\begin{ffcode*}{fontsize=\scriptsize}
<garage DE="2">
  <car country="DE">BMW</car>
  <car country="SE">Volvo</car>
  <car country="DE">Audi</car>
  <car country="JP">Toyota</car>
</garage>
\end{ffcode*}
\columnbreak
This is Xembly script:
\begin{ffcode*}{fontsize=\small}
XPATH "/garage";
XATTR "DE", "count(car[country='DE'])";
\end{ffcode*}
Check it out at \href{https://www.xembly.org}{xembly.org}:\par
\pptQR[1in]{https://www.xembly.org}
\end{pptWide}

\plush{}
\plick{\pptHeader{How EOLANG Compiler Works?}}
\plick{
  EO $\xrightarrow{\text{ANTLR}}$ XMIR
  \pptPin{
    ``XMIR'' stands for XML Intermediate Representation\\[18pt]
    ``Saxon'' is a Java library doing XSL transformations\\[18pt]
    ``Xembly'' is our Java library for imperative XML manipulations
  }
}
\plick{XMIR $\xrightarrow{\text{Saxon}}$ XMIR$_1$}
\plick{XMIR$_1$ $\xrightarrow{\text{Xembly}}$ XMIR$_2$}
\plick{$\dots$}
\plick{XMIR$_{n > 30}$ $\xrightarrow{\text{Saxon}}$ Java}
\plush{Java $\xrightarrow{\text{javac}}$ Bytecode}

\plush{
  \pptHeader{Want to Follow EOLANG Development?}
  Join our Telegram group: @polystat\_org\par
  \pptQR{https://t.me/polystat_org}
}

\pptHeader{Example: Detecting Duplicate Attributes}
\begin{pptWide}{2}
This is EO program:
\begin{ffcode*}{fontsize=\scriptsize}
[] > app
  "Hello" > x
  42 > x
\end{ffcode*}
This is XMIR generated by ANTLR4:
\begin{ffcode*}{fontsize=\scriptsize}
<program>
  <objects>
    <o name="app">
      <o name="x">Hello</o>
      <o name="x">42</o>
    </o>
  </objects>
</program>
\end{ffcode*}
\columnbreak
This is the XSL to check for duplicates:
\begin{ffcode*}{fontsize=\scriptsize}
<xsl:stylesheet>
  <xsl:template match="o[
    count(distinct-values(o/@name)) !=
    count(o/@name)]">
    <!-- add 'error' element -->
  </xsl:template>
</xsl:stylesheet>
\end{ffcode*}
This is XMIR after the XSL:
\begin{ffcode*}{fontsize=\scriptsize}
<program>
  <errors>
    <error>Attribute 'x' is duplicated</error>
  </errors>
  <objects>...</objects>
</program>
\end{ffcode*}
\end{pptWide}

\pptHeader{Simple Immutable Train}
\begin{ffcode*}{fontsize=\small}
interface Shift {
  XML apply(XML input);
}

interface Train {
  Train with(Shift extra);
}

var train = new TrDefault();
var xsl = new XSLDocument("<xsl:stylesheet>...");
var shift = new StXSL(xsl);
train = train.with(shift);
var before = new XMLDocument("<garage>...");
var after = new Xsline(train).pass(before);
\end{ffcode*}

\plush{}
\pptHeader{Decorators of Shift}
\begin{ffcode*}{fontsize=\footnotesize}
var train = new TrDefault()
  .with(new StXSL("<xsl:stylesheet>..."))
  .with(new StClasspath("/org/eolang/parser/dups.xsl"))
  .with(new StEndless(new StXSL("...")))
  .with(new StLogged(new StXSL("...")))
  .with(new StFast(new StXSL("...")))
  .with(new StFast(new StLogged(new StXSL("..."))))
  .with(
    new StFast(
      new StLogged(
        new StEndless(
          new StXSL("...")
        )
      )
    )
  )
\end{ffcode*}

\pptHeader{Reducing Noise: Train Decorators}
\begin{pptWide}{2}
\begin{ffcode*}{fontsize=\footnotesize}
var train = new TrDefault()
  .with(
    new StFast(
      new StLogged(
        new StClasspath("/org/eo...")
      )
    )
  )
  .with(
    new StFast(
      new StLogged(
        new StXSL("<xsl:styl...")
      )
    )
  );
\end{ffcode*}
\columnbreak
\begin{ffcode*}{fontsize=\small}
var t1 = new TrFast(
  new TrLogged(
    new TrDefault()
  )
);
var train = t1
  .with(new StClasspath("/org/eo..."))
  .with(new StXSL("<xsl:styl..."));
\end{ffcode*}
\end{pptWide}

\plush{}
\pptHeader{Reducing Noise Even More: Generics}
\begin{pptWide}{2}
\begin{ffcode*}{fontsize=\footnotesize}
var t1 = new TrFast(
  new TrLogged(
    new TrDefault()
  )
);
var train = t1
  .with(new StClasspath("add-refs.xsl"))
  .with(new StClasspath("duplicates.xsl"))
  .with(new StClasspath("catch.xsl"))
  .with(new StClasspath("no-names.xsl"));
\end{ffcode*}
\columnbreak
\begin{ffcode*}{fontsize=\small}
Train<Shift> t1 = new TrDefault();
Train<String> t2 = new TrClasspath(t1)
  .with("add-refs.xsl")
  .with("duplicates.xsl")
  .with("catch.xsl")
  .with("no-names.xsl");
Train<Shift> t3 = t2.back();
\end{ffcode*}
\end{pptWide}

\plush{}
\pptHeader{Backward Transition of Types}
\begin{pptWide}{2}
\begin{ffcode*}{fontsize=\footnotesize}
Train<Shift> t1 = new TrDefault();
Train<String> t2 = new TrClasspath(t1)
  .with("add-refs.xsl")
  .with("duplicates.xsl")
  .with("catch.xsl")
  .with("no-names.xsl");
Train<Shift> t3 = t2.back();
\end{ffcode*}
\columnbreak
\begin{ffcode*}{fontsize=\small}
interface Train<T> {
  Train<T> with(T extra);
  interface Temporary<X> {
    Train<X> back();
  }
}

class TrClasspath implements
  Train<String>,
  Train.Temporary<Shift> {
  // ...
}
\end{ffcode*}
\end{pptWide}

\plush{}
\pptHeader{Reducing Noise Completely: Bulk Ops}
\begin{pptWide}{2}
\begin{ffcode*}{fontsize=\footnotesize}
Train<Shift> t1 = new TrDefault();
Train<String> t2 = new TrClasspath(t1)
Train<Shift, TrClasspath> t3 =
  new TrBulk<>(t2);
Train<Shift> t4 = t3.with(
  Arrays.asList(
    "add-refs.xsl",
    "duplicates.xsl",
    "catch.xsl",
    "no-names.xsl"
  )
).back().back();
Train<Shift> t3 = t2.back();
\end{ffcode*}
\columnbreak
\begin{ffcode*}{fontsize=\small}
class TrBulk<T, R extends Train<T>>
  implements Train<Iterable<T>>,
  Train.Temporary<T> {
  private final R origin;
  @Override
  TrBulk<T, R> with(Iterable<T> shifts) {
    // return new 'origin' + shifts
  }
  @Override
  R back() {
    return this.origin;
  }
}
\end{ffcode*}
\end{pptWide}

\plush{}
\pptHeader{Secondary Ctors to Help}
\begin{pptWide}{2}
\begin{ffcode*}{fontsize=\footnotesize}
var t = new TrBulk<>(
  new TrClasspath(
    new TrFast(
      new TrLogged(
        new TrDefault()
      )
    )
)
var train = t
  .with("add-refs.xsl")
  .with("duplicates.xsl")
  .with("catch.xsl")
  .back()
  .back();
\end{ffcode*}
\columnbreak
\begin{ffcode*}{fontsize=\small}
Train<Shift> t = new TrBulk<>(
  new TrClasspath(
    new TrFast(
      new TrLogged(
        new TrDefault()
      )
    )
  ),
  "add-refs.xsl",
  "duplicates.xsl",
  "catch.xsl"
).back().back();
\end{ffcode*}
\end{pptWide}

\plush{}
\pptHeader{Real Example from EO Sources}
\begin{pptWide}{2}
\setstretch{0.9}
\begin{ffcode*}{fontsize=\scriptsize}
private static final String[] SHEETS = {
  "/org/eolang/parser/errors/not-empty-atoms.xsl",
  "/org/eolang/parser/errors/middle-varargs.xsl",
  "/org/eolang/parser/errors/duplicate-names.xsl",
  "/org/eolang/parser/errors/many-free-attributes.xsl",
  "/org/eolang/parser/errors/broken-aliases.xsl",
  "/org/eolang/parser/errors/duplicate-aliases.xsl",
  "/org/eolang/parser/errors/global-nonames.xsl",
  "/org/eolang/parser/errors/same-line-names.xsl",
  "/org/eolang/parser/errors/self-naming.xsl",
  "/org/eolang/parser/add-refs.xsl",
  "/org/eolang/parser/wrap-method-calls.xsl",
  "/org/eolang/parser/vars-float-up.xsl",
  "/org/eolang/parser/add-refs.xsl",
  "/org/eolang/parser/warnings/unsorted-metas.xsl",
  "/org/eolang/parser/expand-aliases.xsl",
  "/org/eolang/parser/resolve-aliases.xsl",
  "/org/eolang/parser/synthetic-references.xsl",
  "/org/eolang/parser/add-default-package.xsl",
  // ... and more
};
\end{ffcode*}
\columnbreak
\begin{ffcode*}{fontsize=\scriptsize}
new TrBulk<>(
  new TrClasspath<>(
    new TrLambda(
      new TrFast(
        new TrLogged(
          new TrDefault<>()
        )
      ),
      shift -> new StAfter(
        shift,
        new StLambda(
          shift::uid,
          (pos, xml) -> EACH.with("step", pos)
            .with("sheet", shift.uid())
            .transform(xml)
        )
      )
    ),
    SHEETS
  )
).back().back()
\end{ffcode*}
\end{pptWide}

\plush{
  \pptHeader{Check It Out on GitHub:}
  \url{https://github.com/yegor256/xsline}\par
  \pptQR{https://github.com/yegor256/xsline}\par
  ... and give it a star! :)
}

\end{document}
