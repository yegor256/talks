% SPDX-FileCopyrightText: Copyright (c) 2023 Yegor Bugayenko
% SPDX-License-Identifier: MIT

\documentclass{article}
\usepackage{../../lecture-notes/notes}
\usepackage{svg}
\usepackage{tikz}
  \usetikzlibrary{shadows}

\graphicspath{{../../faces/}}

\newcommand*{\thetitle}{Enemy at the Gate: AI}
\newcommand*{\thesubtitle}{What is the weapon to defend our code?}
\newcommand*{\theauthor}{Yegor Bugayenko}

\renewcommand\ul[1]{\textcolor{orange}{#1}}

\begin{document}

\pptLeft{October 29th, 2025}
\pptRight{@yegor256}

\definecolor{innoBackground}{HTML}{6C717B}
\pagecolor{innoBackground}
\lnPitch{
  \pptTitle{\color{black}{\thetitle}}{\color{black}{\thesubtitle}}\par
  {\color{white}\scshape \theauthor}
  \newline
  {\color{white}\small Zerocracy\par}
}
\newpage
\pagecolor{white}

\lnPitch{\pptChapter[Threats]{What Are the Threats?}}

\lnPitch{\pptSection[1.Defects]{1. Defects a.k.a Bugs}}

\lnQuote
  [Steve McConnell]
  {steve-mcconnell}
  {Microsoft experiences about \ul{10–20 defects per 1000 lines of code} during in-house testing and 0.5 defects per 1000 lines of code in released product.}
  {mcconnell2004code}

\lnPitch{\pptBanner{A Few Open Source Repositories of Our Time}
  {\ttfamily\small\begin{tabular}{llrr>{\color{orange}}r}
  \toprule
  Project & Stack & KLoC & Issues & I/KLoC \\
  \midrule
  \href{https://github.com/torvalds/kernel}{kernel} & C & 27,300 & 220,600 & 8.1 \\
  \href{https://github.com/tensorflow/tensorflow}{tensorflow} & C++ & 3,600 & 40,200 & 11.2 \\
  \href{https://github.com/flutter/flutter}{flutter} & Dart & 2,100 & 103,000 & 49.0 \\
  \href{https://github.com/rust-lang/rust}{rust} & Rust & 2,100 & 56,500 & 26.9 \\
  \href{https://github.com/microsoft/vscode}{vscode} & TypeScript & 1,500 & 210,000 & 140.0 \\
  \href{https://github.com/apache/spark}{spark} & Java & 1,400 & 53,750 & 38.4 \\
  \href{https://github.com/apache/kafka}{kafka} & Java & 980 & 19,800 & 20.2 \\
  \href{https://github.com/angular/angular}{angular} & TypeScript & 800 & 28,700 & 35.9 \\
  \href{https://github.com/google/guava}{guava} & Java & 346 & 3,600 & 10.4 \\
  \bottomrule
  \end{tabular}}}

\lnPitch{\pptSection[2.Holes]{2. Security Holes}}

\lnQuote
  {gary-mcgraw}
  {A central and critical aspect of the computer \ul{security problem} is a software problem. Software defects with security ramifications—including implementation bugs such as buffer overflows and design flaws such as inconsistent error handling—promise to be with us \ul{for years}.}
  {mcgraw2006software}

\lnPitch{\pptSection[3.Dirt]{3. Complexity, Duplication, and Smells}}

\lnQuote
  {vladimir-khorikov}
  {The \ul{simpler} your solution is, the \ul{better} you are as a software developer. Most software developers can write code that works. Creating code that works \ul{and} is as simple as possible --- that is the true challenge.}
  {khorikov2015kiss}

\lnQuote
  [Robert C. Martin]
  {robert-martin}
  {Duplication is the \ul{primary enemy} of a well-designed system.}
  {martin2008clean}

\lnQuote
  [Mark Seemann]
  {mark-seemann}
  {I like the gardening metaphor's emphasis on activities that combat disorder. Just as you must prune and weed a garden, you must refactor and pay off \ul{technical debt} in your code bases.}
  {seemann2021code}

\lnPitch{\pptChapter[LLM]{Why Now It's Getting Worse?}}

\lnPitch{\pptSection[1.Quality]{1. False Sense of Quality}}

\lnQuote
  [Chun Jie Chong]
  {chun-jie-chong}
  {Quality-wise, we found that LLM generates bare-bones code that lacks defensive programming constructs, and is typically \ul{more complex} (per line of code) compared to human-written code. Fuzzing has revealed that LLM-generated code is more \ul{prone to hangs and crashes} than human-written code.}
  {chong2024artificial}

\lnQuote
  [Altaf Allah Abbassi]
  {altaf-allah-abbassi}
  {33.54\% of the studied sample exhibited multiple \ul{inefficiencies}, indicating that inefficiencies in LLM-generated code are diverse and interconnected. Our findings highlight a \ul{critical gap} in LLMs' capability to generate correct, optimized, and high-quality code.}
  {abbassi2025taxonomy}

\lnQuote
  [Adam Alami]
  {adam-alami}
  {Introduction of an LLM into the social system of code review causes a \ul{disruption} to the inherent social dynamics of the process and to the transition of \ul{accountability} from individual to collective.}
  {alami2025accountability}

\lnPitch{\pptSection[2.Influence]{2. Bad Influence}}

\lnQuote
  [Xuanhua Shi]
  {xuanhua-shi}
  {The coding style of human-written code may be influenced by LLMs: they may not only mirror existing norms but also subtly reshape them, gradually pushing human developers toward greater stylistic alignment with \ul{LLM-preferred conventions}.}
  {xu2025transformed}

\lnPitch{\pptSection[Leak]{3. Accidental Leakage of Data}}

\lnPitch{\pptSection[Pressure]{4. Management Pressure}}

\lnPitch{\pptChapter[Defense]{How Can We Defend Ourselves?}}

\lnPitch{\pptSection[Standards]{1. Write Up Quality Standards}}

\lnPitch{\pptSection[Architecture]{2. Explain the Architecture}}

\lnPitch{\pptSection[Tests]{3. Create More and Better Tests}}

\lnPitch{\pptSection[Reviews]{4. Practice Stricter Code Reviews}}

\plush{
  \begin{multicols}{2}
  \pptBanner{1)~Subscribe:}\par
  \qrcode[height=8em]{https://t.me/yegor256news}\par
  \href{https://t.me/yegor256news}{\texttt{@yegor256news}}
  \par\columnbreak\par
  \pptBanner{2)~Join:}\par
  Join me in all your favorite social networks, including GitHub and Twitter:\par
  {\Huge\texttt{@yegor256}}
  \end{multicols}}

\end{document}
