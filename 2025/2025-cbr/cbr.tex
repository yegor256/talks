% SPDX-FileCopyrightText: Copyright (c) 2023 Yegor Bugayenko
% SPDX-License-Identifier: MIT

\documentclass{article}
\usepackage{../../lecture-notes/notes}
\usepackage{svg}
\usepackage{tikz}
  \usetikzlibrary{shadows}

\graphicspath{{../../faces/}}

\newcommand*{\thetitle}{We Are at a War!}
\newcommand*{\thesubtitle}{AI vs Humans}
\newcommand*{\theauthor}{Yegor Bugayenko}

\newcommand\hlt[1]{\textcolor{orange}{\textbf{#1}}}

\begin{document}

\pptLeft{October 29th, 2025}
\pptRight{@yegor256}

\definecolor{innoBackground}{HTML}{6C717B}
\pagecolor{innoBackground}
\plush{
  \begin{pptMiddle}
    \pptTitle{\color{black}{\thetitle}}{\color{black}{\thesubtitle}}\par
    {\color{white}\scshape \theauthor}
    \newline
    {\color{white}\small Zerocracy\par}
  \end{pptMiddle}
}
\newpage
\pagecolor{white}

\plush{\pptChapter[Threats]{What Are the Threats?}}

\plush{\pptSection[Defects]{1. Defects a.k.a Bugs}}

\lnQuote
  [Steve McConnell]
  {steve-mcconnell}
  {Microsoft experiences about \hlt{10–20 defects per 1000 lines of code} during in-house testing and 0.5 defects per 1000 lines of code in released product.}
  {mcconnell2004code}

\lnPitch{\pptBanner{A Few Open Source Repositories of Our Time}
  {\ttfamily\small\begin{tabular}{llrr>{\color{orange}}r}
  \toprule
  Project & Stack & KLoC & Issues & I/KLoC \\
  \midrule
  \href{https://github.com/torvalds/kernel}{kernel} & C & 27,300 & 220,600 & 8.1 \\
  \href{https://github.com/tensorflow/tensorflow}{tensorflow} & C++ & 3,600 & 40,200 & 11.2 \\
  \href{https://github.com/flutter/flutter}{flutter} & Dart & 2,100 & 103,000 & 49.0 \\
  \href{https://github.com/rust-lang/rust}{rust} & Rust & 2,100 & 56,500 & 26.9 \\
  \href{https://github.com/microsoft/vscode}{vscode} & TypeScript & 1,500 & 210,000 & 140.0 \\
  \href{https://github.com/apache/spark}{spark} & Java & 1,400 & 53,750 & 38.4 \\
  \href{https://github.com/apache/kafka}{kafka} & Java & 980 & 19,800 & 20.2 \\
  \href{https://github.com/angular/angular}{angular} & TypeScript & 800 & 28,700 & 35.9 \\
  \href{https://github.com/google/guava}{guava} & Java & 346 & 3,600 & 10.4 \\
  \bottomrule
  \end{tabular}}}

\plush{\pptSection[Holes]{2. Security Holes}}

\plush{\pptSection[Dirty Code]{3. Dirty Code}}
  % inconsistent (dirty) code: duplicated, hard to read, too complex, not reusable

\plush{\pptChapter[LLM]{Why Now It's Getting Worse?}}

\lnQuote
  [Chun Jie Chong]
  {chun-jie-chong}
  {Quality-wise, we found that LLM generates bare-bones code that lacks defensive programming constructs, and is typically \hlt{more complex} (per line of code) compared to human-written code. Fuzzing has revealed that LLM-generated code is more \hlt{prone to hangs and crashes} than human-written code.}
  {chong2024artificial}

  % false sense of security means less scrutiny, more silent bugs slipping through
  % Repetition of existing errors: once LLM finds it in the source code it copies
  % LLM doesn't know what code it copies -- violation of licenses
  % Invisible leak of data to LLM provider: passwords or private Bitcoin keys
  % management pressure

\plush{\pptChapter[Defense]{How Can We Defend Ourselves?}}

  % focus on quality standards, in written way
  % explain the architecture - it matters for AI
  % focus on reviews!
  % more tests!
  % traceability matters much more -- who wrote this code?
  % separate AI written code from others

\plush{
  \begin{multicols}{2}
  \pptBanner{1)~Subscribe:}\par
  \qrcode[height=8em]{https://t.me/yegor256news}\par
  \href{https://t.me/yegor256news}{\texttt{@yegor256news}}
  \par\columnbreak\par
  \pptBanner{2)~Join:}\par
  Join me in all your favorite social networks, including GitHub and Twitter:\par
  {\Huge\texttt{@yegor256}}
  \end{multicols}}

\end{document}
