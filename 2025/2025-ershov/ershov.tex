% SPDX-FileCopyrightText: Copyright (c) 2023 Yegor Bugayenko
% SPDX-License-Identifier: MIT

\documentclass{article}
\usepackage{../../lecture-notes/notes}
\usepackage[russian,english]{babel}
\usepackage{svg}
\usepackage{bibentry}
\usepackage{fancyvrb}
\usepackage{tikz}
  \usetikzlibrary{shadows}

\graphicspath{{../../faces/}}

\newcommand*{\thetitle}{Rise and Fall of OOP}
\newcommand*{\thesubtitle}{{\selectlanguage{russian}Каковы перспективы?}}
\newcommand*{\theauthor}{Yegor Bugayenko}
\newcommand\hlt[1]{\textcolor{orange}{#1}}

\newenvironment{lnSnippet}[1][lnSnippet.txt]
  {\VerbatimEnvironment\begin{VerbatimOut}{#1}}
  {\end{VerbatimOut}}

\begin{document}

\pptLeft{%
  \includesvg[height=2em]{../../lecture-notes/logos/ngu.svg}\\
  April 11th, 2025}
\pptRight{@yegor256}

\lnPitch{
  \begin{pptMiddle}
    \includesvg[height=4em]{../../lecture-notes/logos/ngu.svg}
    \pptTitle{\thetitle}{\thesubtitle}\par
    {\scshape \theauthor}
    \newline
    {\small Huawei\par}
  \end{pptMiddle}
}
\newpage
\pagecolor{white}

% \pptToc

\pptPin{\includesvg[height=5em]{../../painofoop/00-syllabus/cactus.svg}}
\lnPitch{
  \pptBanner{Why Me?}
  \begin{enumerate}
  \item \href{https://www.yegor256.com/elegant-objects.html}{\textbf{Elegant Objects}} books: No.1 in \href{https://www.goodreads.com/shelf/show/object-oriented-programming}{Goodreads} OOP shelf
  \item \href{https://www.takes.org}{\textbf{Takes}}, object-oriented Java web framework: 800\(\star\)
  \item \href{https://www.eolang.org}{\textbf{EOLANG}}, pure object-oriented language: 1.2K\(\star\)
  \end{enumerate}}

\lnPitch{
  \pptBanner{What's the Structure of the Talk?}
  \begin{enumerate}
  \item OOP is great!
  \item Maybe OOP is not so great?
  \item How to make it great?
  \end{enumerate}}

\lnPitch{\pptBanner[red]{\HUGE Warning!}}

\lnPitch{\pptChapter[Languages]{Three Languages}}

\lnPitch{
  \pptSection[SIMULA]{1. Why Did They Invent Objects?}
  \begin{multicols}{2}
  \pptPic{.95}{../../painofoop/01-algorithms/dahl-and-nygaard.jpg}\par
  {\scriptsize In 1965, Ole-Johan Dahl and Kristen Nygaard in Norwegian Computing Center in Oslo.\par}
  \par\columnbreak\par
  ``They formalized the concept of \hlt{simulation} into a mathematical language that would be easy for a computer to understand and easier for analysts to write. \textbf{SIMULA~67} was the first object-oriented programming language.''
  \lnSource{martin2025we}
  \end{multicols}}

\lnPitch{
  \pptSection[Smalltalk]{2. OOP Was a Teaching Experiment}
  \begin{multicols}{2}
  \pptPic{0.95}{../../painofoop/01-algorithms/smalltalk-guys.jpg}\par
  {\scriptsize In 1970s, Alan Kay and others, in Xerox PARC.\par}
  \par\columnbreak\par
  ``\textbf{Smalltalk-80}, took Simula's object-oriented paradigm to its natural conclusion by making everything in the language an instance of a class --- \hlt{everything is an object}.''
  \lnSource{booch1994object}
  \end{multicols}}

\lnPitch{
  \pptSection[C++]{3. Eventually, OOP Became a Real Deal}
  \begin{multicols}{2}
  \pptPic{0.95}{../../painofoop/01-algorithms/bjarne-in-1985.jpg}\par
  {\scriptsize In 1985, Bjarne Stroustrup, Danish computer scientist.\par}
  \par\columnbreak\par
  ``\textbf{C++} was designed to provide Simula's facilities for program organization together with C's efficiency and flexibility for systems programming. It was intended to deliver that to \hlt{real projects} within half a year of the idea. It succeeded.''
  \lnSource{stroustrup1994design}
  \end{multicols}}

\lnPitch{\pptChapter[Intent]{Objects Are Abstractions}}

\lnPitch{
  \pptSection[Proxies]{Objects Are Proxies for Real Things}
  \begin{multicols}{2}
  \includegraphics[width=.95\linewidth]{../../painofoop/01-algorithms/abstraction.png}
  \par\columnbreak\par
  ``Object-oriented design is first concerned with \hlt{entities}---things. These things may be tangible objects such as traffic lights, chairs, or airplanes. From a design perspective, objects \hlt{model} the entities in the application domain.''
  \lnSource{korson1990understanding}
  \end{multicols}}

\lnQuote
  [Hemant Pande]
  {hemant-pande}
  {Dynamic binding and polymorphism, which only exist in objects, support \hlt{abstraction} and differentiate object-orientation from the \hlt{imperative} programming paradigm.}
  {pande1996data}

\begin{lnSnippet}[animals.java]
interface Fruit
  void calories();
class (*@\textcolor{orange}{Plum}@*)
  @Override
  void calories() { ... }
class (*@\textcolor{orange}{Apple}@*)
  @Override
  void calories() { ... }

void cook(Fruit a)
  var c = a(*@\textcolor{orange}{.calories}@*)();
\end{lnSnippet}
\lnPitch{
  \pptSection[Binding]{1.~Dynamic Dispatch}
  \begin{multicols}{2}
  {\small\ffinput{animals.java}}
  \par\columnbreak\par
  ``The most important language feature that supports abstraction is the dynamic dispatch of methods based on the \hlt{run-time type} of an object.''
  \lnSource{bacon1996fast}
  \end{multicols}}

\lnPitch{
  \pptSection[Encapsulation]{2.~Encapsulation}
  \begin{multicols}{2}
  \includegraphics[height=1.5in]{../../painofoop/01-algorithms/apple.jpg}
  \includegraphics[height=1.2in]{plum.jpg}
  \par
  \begin{tabular}{ll}
  Weight: & 120g \\
  Calories: & 94.6 \\
  Price: & \$0.99 \\
  \end{tabular}
  \par\columnbreak\par
  ``Once an implementation is selected, it should be treated as a \hlt{secret} of the abstraction and hidden from most \hlt{clients}.''
  \lnSource{booch1994object}
  \end{multicols}}

\lnQuote
  [Bertrand Meyer]
  {bertrand-meyer}
  {He who moves the most features the highest, as a result of \hlt{discovering} higher-level abstractions, and along the way merges the most tokens, as a result of discovering commonalities, wins.}
  {meyer1997object}

\lnPitch{\pptChapter[Slow]{However, OOP Is Slow}}

\lnQuote
  [Craig Chambers]
  {craig-chambers}
  {OO languages contain a number of features that make programs \hlt{easier to write} but \hlt{slower to run}. Pure OO languages use message passing for all computation, avoiding built-in operators and control structures. All computation, even low-level operations like variable accessing, arithmetic, and array indexing, is performed by \hlt{sending messages} to objects.}
  {chambers1991making}

\begin{lnSnippet}[max-static.java]
x = a (*@\textcolor{orange}{+}@*) b;
\end{lnSnippet}
\begin{lnSnippet}[max-method.java]
x = a(*@\textcolor{orange}{.plus}@*)(b);
\end{lnSnippet}
\begin{lnSnippet}[max-object.java]
x = (*@\textcolor{orange}{new Plus}@*)(a, b).get();
\end{lnSnippet}
\lnPitch{
  \begin{multicols}{2}
  Instead of doing this, in Java:
  {\small\ffinput{max-static.java}}
  We should do this:
  {\small\ffinput{max-method.java}}
  Or maybe even this:
  {\small\ffinput{max-object.java}}
  \par\columnbreak\par
  ``Pure object-oriented languages exacerbate the performance problem caused by dynamic dispatch because \hlt{every} operation involves a dynamically dispatched message send.''
  \lnSource{holzle1996reconciling}
  \end{multicols}}

\lnQuote
  [Jeff Piper]
  {jeff-piper}
  {We observe that there is a substantial cost associated with a \hlt{full object-oriented design} in scientific programs, and that the current compilers and VMs have a room for substantial improvements in this area.}
  {budimlic1999cost}
\lnPitch{
  \pptSection[1.2]{JDK 1.2 Was Very Slow}
  \includegraphics[width=.8\linewidth]{bulimic.png}\par
  \lnSource{budimlic1999cost}}
\lnPitch{
  \pptSection[21]{JDK 21 Is not Much Better}
  {\small\begin{tabularx}{\linewidth}{lX>{\ttfamily}r>{\ttfamily}r>{\ttfamily\arraybackslash}r}
  \toprule
  & & \multicolumn{2}{c}{CPU mInstructions} & \\
  Language & Compiler & {\rmfamily w/functions} & {\rmfamily w/objects} & {\rmfamily Ratio} \\
  \midrule
  C++ & clang 18.1.3 & 93 & 7,203 & \textcolor{orange}{76x} \\
  Java & javac 21.0.4 & 41 & 4,589 & \textcolor{orange}{109x} \\
  C\# & 8.0.108 & 36 & 5,785 & \textcolor{orange}{157x} \\
  Go & go1.22.2 & 39 & 15,907 & \textcolor{orange}{403x} \\
  Pascal & 3.2.2 & 59 & 32,804 & \textcolor{orange}{555x} \\
  \bottomrule
  \end{tabularx}}\par
  {\scriptsize The same Fibonacci algorithm was implemented in different programming languages using either functions (the third column) or objects (the forth column). Performance data was collected using \ff{perf} Linux tool (Ubuntu 20.04 at Intel Core i7, 16Gb, 3.3GHz), as millions of CPU instructions per a calculation of the 32\(^\text{nd}\) Fibonacci number. Source: \url{https://github.com/yegor256/fibonacci}.\par}}

\lnPitch{\pptChapter[Reasons]{Objects Are Slow for Two Reasons}}

\lnQuote
  [Bruno Dufour]
  {bruno-dufour}
  {\hlt{Virtual method dispatching} and \hlt{on-heap allocating} are the two primary sources of performance inefficiencies in object-oriented programs.}
  {dufour2003dynamic}

\begin{lnSnippet}[streams.java]
var acc = Stream.of(VALUES)
  .map(obj -> (String) obj)
  .map(String::trim)
  .filter(str -> str.length() == 4)
  .map(str -> Long.parseLong(str, 16))
  .mapToLong(num -> num)
  .sum();
\end{lnSnippet}
\begin{lnSnippet}[loop.java]
var acc = 0L;
for (int i = 0; i < VALUES.length; i++) {
  var s = ((String) VALUES[i]).trim();
  if (s.length() != 4) continue;
  acc += Long.parseLong(s, 16);
}
\end{lnSnippet}
\lnPitch{
  \pptSection[Streams]{For Example, Java Stream API vs. Loop}
  \begin{multicols}{2}
  Calculating the sum with the help of Stream API with OpenJDK Zulu~23.30:
  \par\columnbreak\par
  Exactly the same algorithm, but in imperative procedural style:
  \end{multicols}
  \par
  \begin{multicols}{2}
  {\scriptsize\ffinput{streams.java}\par}
  \par\columnbreak\par
  {\scriptsize\ffinput{loop.java}\par}
  \end{multicols}
  \par
  \begin{multicols}{2}
  \textcolor{red}{\textbf{154}} milliseconds.
  \par\columnbreak\par
  \textcolor{green}{\textbf{26}} milliseconds.
  \end{multicols}
  {\scriptsize The \ff{VALUES} contains 10 million strings, on MacBook M2~Pro 3.5~GHz.\par}}

\lnPitch{\pptChapter[Compilers]{Compilers May Help}}

\lnQuote
  [Zoran Budimli{\'c},]
  {zoran-budimlic}
  {Although Java implementations have been made great strides, they still fall short on programs that use the full power of Java’s object-oriented features. Ideally, future compiler technologies will be able to \hlt{automatically transform} the [OO style code] into something that approaches the [procedural style] in performance.}
  {budimlic1999cost}

\lnPitch{\pptChapter[ALGOL]{Instead: ALGOL in Java Syntax}}

\lnQuote
  [Robert Martin]
  {robert-martin}
  {After all, we program in Java, or C\#, or Ruby, and we use object-oriented design. True---and yet the code is still just an assemblage of \hlt{sequence}, \hlt{selection}, and \hlt{iteration}, just as it was back in the 1960s and 1950s.}
  {martin2017clean}

\begin{lnSnippet}[static.java]
var bytes = "Hello".getBytes();
Files.(*@\textcolor{orange}{write}@*)(path, bytes);
\end{lnSnippet}
\begin{lnSnippet}[no-static.java]
"Hello".toBytes().(*@\textcolor{orange}{saveTo}@*)(path);
\end{lnSnippet}
\lnPitch{
  \pptSection[Static]{Issue \#1: Egocentric Static Methods}
  \begin{multicols}{2}
  In Java, we do this:
  {\small\ffinput{static.java}\par}
  Instead of this:
  {\small\ffinput{no-static.java}\par}
  \par\columnbreak\par
  ``Data manipulated and moved by active procedures are passive and helpless. Procedures tend to be \hlt{egocentric} and think that all data, its definitions and its values, are properly determined by the procedure currently using that data. The result is data that is \hlt{inconsistent} in definition and value.''
  \lnSource{west2004object}
  \end{multicols}}

\begin{lnSnippet}[with-null.java]
var u = getUser(42);
(*@\textcolor{orange}{if (u == null)}@*)
  return "Not found";
var n = u.getName();
\end{lnSnippet}
\begin{lnSnippet}[without-null.java]
var n = user(42).(*@\textcolor{orange}{name}@*)();
\end{lnSnippet}
\lnPitch{
  \pptSection[NULL]{Issue \#2: NULL, the Billion Dollar Mistake}
  \begin{multicols}{2}
  We do this:
  {\small\ffinput{with-null.java}\par}
  Instead of this:
  {\small\ffinput{without-null.java}\par}
  \par\columnbreak\par
  ``This led me to suggest that the null value is a member of every type, and a \hlt{null check is required} on every use of that reference variable, and it may be perhaps a billion dollar mistake.''
  \lnSource{hoare2009null}
  \end{multicols}}

\lnPitch{
  \includegraphics[width=.9\linewidth]{tweet.png}\par
  {\scriptsize BTW, my Twitter: @yegor256}}

\begin{lnSnippet}[inheritance.java]
class Book (*@\textcolor{orange}{extends File}@*)
  void title()
    var t = (*@\textcolor{orange}{this.}@*)read();
    return t.find("title");
\end{lnSnippet}
\begin{lnSnippet}[composition.java]
class Book
  (*@\textcolor{orange}{File file;}@*)
  void title()
    var t = (*@\textcolor{orange}{file.}@*)read();
    return t.find("title");
\end{lnSnippet}
\lnPitch{
  \pptSection[Inheritance]{Issue \#3: Implementation Inheritance}
  \begin{multicols}{2}
  We do this:
  {\small\ffinput{inheritance.java}\par}
  Instead of this:
  {\small\ffinput{composition.java}\par}
  \par\columnbreak\par
  ``Favoring object composition over class inheritance helps you keep each class encapsulated and focused on one task. Your classes and class hierarchies will remain small and will be less likely to grow into \hlt{unmanageable monsters}.''
  \lnSource{gamma1994design}
  \end{multicols}}

\begin{lnSnippet}[casting.java]
cook(Fruit f)
  var c = nil;
  if (f (*@\textcolor{orange}{instanceof}@*) Apple a)
    c = a.color();
  else
    c = "unknown"
\end{lnSnippet}
\begin{lnSnippet}[no-casting.java]
cook(Fruit f)
  var c = f(*@\textcolor{orange}{.color()}@*);
\end{lnSnippet}
\lnPitch{
  \pptSection[Casting]{Issue \#4: Type Casting \& Reflection}
  \begin{multicols}{2}
  In Java 14, we do this:
  {\small\ffinput{casting.java}\par}
  Instead of this:
  {\small\ffinput{no-casting.java}\par}
  \par\columnbreak\par
  ``Type casting turns off your complier's ability to check for type mismatches and therefore creates a \hlt{hole} in your defensive-programming armor. Try to avoid type casts as much as you can.''
  \lnSource{mcconnell2004codecomplete}
  \end{multicols}}

\begin{lnSnippet}[mutable.java]
var c = new Circle();
c.(*@\textcolor{orange}{setRadius}@*)(42.0);
c.(*@\textcolor{orange}{setColor}@*)("blue");
\end{lnSnippet}
\begin{lnSnippet}[immutable.java]
var c = new Circle()
  .(*@\textcolor{orange}{resize}@*)(42.0)
  .(*@\textcolor{orange}{repaint}@*)("blue");
\end{lnSnippet}
\lnPitch{
  \pptSection[Mutability]{Issue \#5: Mutable State}
  \begin{multicols}{2}
  We do this:
  {\small\ffinput{mutable.java}\par}
  Instead of this:
  {\small\ffinput{immutable.java}\par}
  \par\columnbreak\par
  ``Immutable classes are \hlt{easier to design}, implement, and use than mutable classes. They are less prone to error and are more secure.''
  \lnSource{bloch2008effective}
  \end{multicols}}

\lnQuote
  [David West]
  {david-west}
  {The contemporary mainstream understanding of objects (which is \hlt{not behavioral}) is but a pale shadow of the original idea and anti-ethical to the original intent.}
  {west2004object}

\lnPitch{\pptChapter[Disaster]{OOP Is a Disaster}}

\lnQuote
  [Edsger Dijkstra]
  {edsger-dijkstra}
  {Object oriented programs are offered as alternatives to correct ones... Object-oriented programming is an \hlt{exceptionally bad idea} which could only have originated in California.}
  {crawford1989}

\lnQuote
  [Alan Kay]
  {alan-kay}
  {I made up the term `object-oriented,' and I can tell you I didn't have C++ in mind.}
  {kay97keynote}

\lnQuote
  [Linus Torvalds]
  {linus-torvalds}
  {C++ is a \hlt{horrible language}$\dots$ C++ leads to really, really bad design choices$\dots$ In other words, the only way to do good, efficient, and system-level and portable C++ ends up to limit yourself to all the things that are basically available in C.}
  {schindler2007}

\lnQuote
  [Jeff Atwood]
  {jeff-atwood}
  {OO seems to bring at least as \hlt{many problems} to the table as it solves.}
  {atwood2007}

\lnQuote
  [Rich Hickey]
  {rich-hickey}
  {I think that large objected-oriented programs struggle with increasing complexity as you build this large object graph of \hlt{mutable objects}. You know, trying to understand and keep in your mind what will happen when you call a method and what will the side effects be.}
  {hickey2010}

\lnPitch{
  \pptBanner{Design (Anti-)Patterns:}
  \begin{itemize}
  \setlength\itemsep{0em}
  \item Singleton
  \item Data Transfer Object (DTO)
  \item Factory Method
  \item Object Relational Mapping (ORM)
  \item Model View Controller (MVC)
  \end{itemize}}

\lnPitch{\pptChapter[Pure]{Let's Make OOP Great Again}}

\lnPitch{
  \pptBanner{``Pure'' Objects Are Possible:}
  \begin{itemize}
  \setlength\itemsep{0em}
  \item No static methods or attributes
  \item No mutability of state
  \item No implementation inheritance
  \item No NULL references
  \item No reflection, no annotations, no classes
  \end{itemize}
  In some languages most objects are pure,
  for example in \hlt{Self}, \hlt{Io}, and \hlt{EO}.}

\lnPitch{\pptChapter[Fast]{Can We Make Them Fast?}}

\lnThought{We can design a \textcolor{orange}{strict} and "pure" language—without procedures, null references, or mutability.}

\lnThought{Because objects are less complex, the compiler can \textcolor{orange}{inline} them at compile time, effectively turning them into procedures.}

\lnThought{We can place more objects on the \textcolor{orange}{stack}, reducing the workload for the memory manager and garbage collector.}

\lnThought{We should build \textcolor{orange}{hardware} with an object-oriented instruction set, including directives like NEW and CALL.}

\lnQuote
  {nature}
  {The underlying concept of the computer hardware design that has remained fundamentally unchanged since the days of von Neumann is in need of \hlt{serious reform}.}
  {beyond2022}

\lnPitch{
  \begin{multicols}{2}
  \qrcode[height=8em]{https://t.me/yegor256news}\par
  \href{https://t.me/yegor256news}{\texttt{@yegor256news}}
  \par\columnbreak\par
  Subscribe to my Telegram channel to stay updated on the latest changes in the EOLANG project and other research projects we're involved in.
  \end{multicols}}

\end{document}
