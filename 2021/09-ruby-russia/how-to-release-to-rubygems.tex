% (The MIT License)
%
% Copyright (c) 2021 Yegor Bugayenko
%
% Permission is hereby granted, free of charge, to any person obtaining a copy
% of this software and associated documentation files (the 'Software'), to deal
% in the Software without restriction, including without limitation the rights
% to use, copy, modify, merge, publish, distribute, sublicense, and/or sell
% copies of the Software, and to permit persons to whom the Software is
% furnished to do so, subject to the following conditions:
%
% The above copyright notice and this permission notice shall be included in all
% copies or substantial portions of the Software.
%
% THE SOFTWARE IS PROVIDED 'AS IS', WITHOUT WARRANTY OF ANY KIND, EXPRESS OR
% IMPLIED, INCLUDING BUT NOT LIMITED TO THE WARRANTIES OF MERCHANTABILITY,
% FITNESS FOR A PARTICULAR PURPOSE AND NONINFRINGEMENT. IN NO EVENT SHALL THE
% AUTHORS OR COPYRIGHT HOLDERS BE LIABLE FOR ANY CLAIM, DAMAGES OR OTHER
% LIABILITY, WHETHER IN AN ACTION OF CONTRACT, TORT OR OTHERWISE, ARISING FROM,
% OUT OF OR IN CONNECTION WITH THE SOFTWARE OR THE USE OR OTHER DEALINGS IN THE
% SOFTWARE.

% \documentclass[landscape]{/code/huawei.cls/huawei}

\documentclass{article}
\usepackage[nocn]{/code/ffcode/ffcode}
\usepackage{/code/ssd16/inno}

\usepackage[fontsize=24pt]{fontsize}
\usepackage{anyfontsize}
\usepackage{/code/ssd16/slides}
\usepackage[static]{/code/clicks/clicks}
\usepackage{changepage}
\usepackage{soul}
\usepackage{amsmath}
\usepackage{setspace}
\newcommand*\theauthor{Yegor Bugayenko}
\newcommand*\thetitle{RubyGems}
\newcommand*\thesubtitle{How to release to it?}
\innoLeft{\thetitle{} \thesubtitle}
\innoRight{@yegor256}
\begin{document}

\plush[2]{
  \innoMiddle{
    \innoTitle{\thetitle}{\thesubtitle}\par
    {\scshape\theauthor}
  }
}

\plick{\innoHeader{My Story}}
\plick{I'm a Java programmer, since Java 1.0 (1996)}
\plick{I met Ruby in 2018 (just three years ago)}
\plick{Since then I released 35 Ruby gems}
\plick{One of them is a cryptocurrency}
\plick{Total 800+ GitHub stars}
\plush[2]{Total 1.4M downloads at RubyGems.org}

\plick{\innoHeader{My/Your Motivation}}
\plick{Pull requests from volunteers}
\plick{Higher quality due to better discipline}
\plick{Publicity}
\plush[6]{Money}

\innoHeader{\ttfamily iri}
\begin{ffcode*}{fontsize=\scriptsize}
require 'iri'
url = Iri.new('http://google.com/')
  .append('find').append('me')
  .add(q: 'books about OOP', limit: 50)
  .del(:q)
  .del('limit')
  .over(q: 'books about tennis', limit: 10)
  .scheme('https')
  .host('localhost')
  .port('443')
  .fragment('page-4')
  .query('a=1&b=2')
  .path('/new/path')
  .cut('/q')
  .to_s
\end{ffcode*}
\flush[3]

\plick{\innoHeader{Steps}}
\plick{1) StackOverflow question}
\plick{2) README prototype}
\plick{3) Telegram groups}
\plick{4) \ttfamily iri.gemspec}
\plick{5) \ttfamily rubygems.yml}
\plick{6) \ttfamily .rultor.yml}
\plush[3]{7) Twitter post + spam in groups}

\plush[2]{
  \innoHeader{StackOverflow Question}
  \innoPic{0.7}{so1}
}

\innoHeader{\ttfamily iri.gemspec}
\begin{innoWide}{2}
\begin{ffcode*}{fontsize=\scriptsize}
require 'English'
Gem::Specification.new do |s|
  s.rubygems_version = '2.5'
  s.required_ruby_version = '>=2.2'
  s.name = 'iri'
  s.version = '0.0.0'
  s.license = 'MIT'
  s.summary = 'Simple Immutable Ruby URI Builder'
  s.description = 'Class Iri helps you build a URI and then modify its \
parts via a simple fluent interface.'
  s.authors = ['Yegor Bugayenko']
  s.email = 'yegor256@gmail.com'
  s.homepage = 'https://github.com/yegor256/iri'
  s.files = `git ls-files`.split(\$RS)
  s.test_files = s.files.grep(%r{^(test)/})
  s.rdoc_options = ['--charset=UTF-8']
  s.extra_rdoc_files = ['README.md']
  s.add_development_dependency 'codecov', '0.2.11'
  s.add_development_dependency 'minitest', '5.11.3'
  s.add_development_dependency 'rake', '12.3.3'
  s.add_development_dependency 'rdoc', '6.3.1'
  s.add_development_dependency 'rubocop', '0.62.0'
  s.add_development_dependency 'rubocop-rspec', '1.31.0'
end
\end{ffcode*}
\end{innoWide}
\flush[3]

\innoHeader{\ttfamily rubygems.yml}
\begin{ffcode*}{fontsize=\scriptsize}
---
:rubygems_api_key: d355d.....b051bfb9ac.....0daac0d
\end{ffcode*}
\flush[1]

\innoHeader{\ttfamily .rultor.yml}
\begin{ffcode*}{fontsize=\scriptsize,highlightlines={14},highlightcolor=green!80}
assets:
  rubygems.yml: yegor256/home#assets/rubygems.yml
install: |$\vert$|-
  bundle install
release:
  script: |$\vert$|-
    bundle exec rake
    rm -rf *.gem
    sed -i "s/0\.0\.0/${tag}/g" iri.gemspec
    git add iri.gemspec
    git commit -m "Version set to ${tag}"
    gem build iri.gemspec
    chmod 0600 ../rubygems.yml
    gem push *.gem --config-file ../rubygems.yml
\end{ffcode*}
\flush[3]

\plush[4]{
  \innoHeader{Relase via Rultor}
  \innoPic{0.5}{github}
}

\plush[1]{
  \innoHeader{Email from Rubygems}
  \innoPic{0.6}{email}
}

\plush[1]{
  \innoHeader{StackOverflow Answer}
  \innoPic{0.7}{so2}
}

\end{document}
