% SPDX-FileCopyrightText: Copyright (c) 2021 Yegor Bugayenko
% SPDX-License-Identifier: MIT

\documentclass{article}
\usepackage[utf8]{inputenc}
\usepackage[nocn]{ffcode}
\usepackage{/code/ssd16/inno}
\usepackage{/code/ssd16/slides}
\usepackage[fontsize=24pt]{fontsize}
\usepackage{anyfontsize}
\usepackage{clicks}
\newcommand\nospell[1]{#1}
\newcommand*\theauthor{Yegor Bugaeynko}
\newcommand*\thetitle{Veil Objects}
\newcommand*\thesubtitle{instead of DTOs}
\innoLeft{\thetitle{} \thesubtitle}
\innoRight{@yegor256}
\begin{document}

\plush{
  \innoMiddle{
    \innoTitle{\thetitle}{\thesubtitle}\par
    {\scshape\theauthor}\par
    \url{https://www.yegor256.com/2020/05/19/veil-objects.html}
  }
}

\clearpage
\innoHeader{Fetch Many Items from PostgreSQL}
\begin{innoWideOne}\footnotesize
\begin{ffcode*}{highlightlines={3},highlightcolor=green!80}
class Books
  def fetch
    @pgsql.exec('SELECT * FROM book')
  end
end

# This is what I get back:
[
  {"id": 1, "author": "David West", "title": "Object Thinking"},
  {"id": 2, "author": "Martin Fowler", "author": "Refactoring"}
]
\end{ffcode*}
\end{innoWideOne}
\url{https://github.com/yegor256/pgql}
\clearpage

\clearpage
\innoHeader{I Need Books, Not Hashes}
\begin{innoWideOne}\footnotesize
\begin{ffcode*}{highlightlines={11},highlightcolor=red!80}
class Book
  def toHTML
    # Here we create HTML from the book,
    # using its 'title' and 'author'
    # properties
  end
end

books = Books.new(real_pgsql_client)
books.fetch do |b|
  puts b.toHTML
end
\end{ffcode*}
\end{innoWideOne}
\clearpage

\clearpage
\innoHeader{With DTO Design Pattern}
\begin{innoWideOne}\footnotesize
\begin{ffcode*}{highlightlines={11-15},highlightcolor=red!80}
class BookDTO
  def initialize(hash)
    @hash = hash
  end
  def title
    @hash[:title]
  end
  def author
    @hash[:author]
  end
  def rename(new_title)
    # Oops, this is not possible, since
    # it's an anemic object, with no
    # connection to the database
  end
end
\end{ffcode*}
\end{innoWideOne}
\clearpage

\clearpage
\innoHeader{Without DTO}
\begin{innoWideOne}\footnotesize
\begin{ffcode*}{highlightlines={10},highlightcolor=red!80}
class Book
  def initialize(pgsql, id)
    @pgsql = pgsql
    @id = id
  end
  def title
    @pgsql.exec('SELECT title FROM book WHERE id=$1', [@id])[0]['title']
  end
  def rename(new_title)
    @pgsql.exec('UPDATE book SET title=$1 book WHERE id=$2', [new_title, @id])
  end
end
\end{ffcode*}
\end{innoWideOne}
\clearpage

\clearpage
\innoHeader{With Veil (Object Decorator)}
\begin{innoWideOne}\footnotesize
\begin{ffcode*}{highlightlines={5},highlightcolor=green!80}
require 'veils'
class Books
  def fetch
    @pgsql.exec('SELECT * FROM book').map do |r|
      Veil.new(
        Book.new(@pgsql, r['id'].to_i),
        title: r['title'],
        author: r['author']
      )
    end
  end
end
\end{ffcode*}
\end{innoWideOne}
\clearpage

\clearpage
\innoHeader{How It Behaves}
\begin{innoWideOne}\footnotesize
\begin{ffcode*}{}
books = Books.new(real_pgsql_client)
books.fetch do |b|
  b.toHTML # taking from the hash in veil
  b.rename('New Name') # the veil is pierced
  b.toHTML # making round-trips to PostgreSQL
end
\end{ffcode*}
\end{innoWideOne}
\clearpage

\clearpage
\innoHeader{Unpiercable}
\begin{innoWideOne}\footnotesize
\begin{ffcode*}{}
require 'veils'
class Books
  def fetch
    @pgsql.exec('SELECT * FROM book').map do |r|
      Unpiercable.new(
        Book.new(@pgsql, r['id'].to_i),
        title: r['title'],
        author: r['author'],
        html: calculate_html(r)
      )
    end
  end
end
\end{ffcode*}
\end{innoWideOne}
\clearpage

\clearpage
\innoHeader{Class 'Veil'}
\begin{innoWide}{2}\scriptsize
\begin{ffcode*}{highlightlines={3},highlightcolor=green!80}
class Veil
  def initialize(origin, methods = {})
    @origin = origin
    @methods = methods
    @pierced = false
  end
  def method_missing(*args)
    method = args[0]
    if @pierced |$\vert\vert$| !@methods.key?(method)
      @pierced = true
      unless @origin.respond_to?(method)
        raise "#{method} is not in #{@origin}"
      end
      if block_given?
        @origin.__send__(*args) do |$\vert$|*a|$\vert$|
          yield(*a)
        end
      else
        @origin.__send__(*args)
      end
    else
      @methods[method]
    end
  end
  def respond_to?(method, inc = false)
    @origin.respond_to?(method, inc) |$\vert\vert$| @methods[method]
  end
  def respond_to_missing?(_m, _i = false)
    true
  end
end
\end{ffcode*}
\end{innoWide}
\clearpage

\plush{
  \innoHeader{Real Sample from Codexia.org}
  \innoPic{0.7}{codexia}
  \url{https://github.com/yegor256/codexia}
}

\end{document}
